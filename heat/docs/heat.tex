\documentclass[12pt]{article}
\usepackage{amstext}
\usepackage{amssymb}
\usepackage{amsmath}
\usepackage[dvips]{epsfig}

\def \fr {\frac}
\def \pd {\partial}


\title{The 2D Heat Equation}

\begin{document}

\section{Parallel computing}

Parallel computing is the wave of the future, and always will be


\section{Introduction}
%        ============

3D heat (diffusion) equation in its simplest form is the IVBP defined by the PDE
%
\[
  \fr{\pd u}{\pd t} \ =\ \fr{\pd^2 u}{\pd x^2} \ +\ \fr{\pd^2 u}{\pd y^2} \ +\ \fr{\pd^2 u}{\pd z^2}
\]
%
defined in the unit cube $0\ \leq\ x,~y,~z\ \leq\ 1$.  Boundary and initial conditions are
%
\begin{eqnarray*}
  u(0,y,z,~t) &=& u(1,y,z,~t) = 0 \\
  u(x,0,z,~t) &=& u(x,1,z,~t) = 0 \\
  u(x,y,0,~t) &=& u(x,y,1,~t) = 0 \\
  u(x,y,0) &=& sin(\pi l x)\ \sin(\pi m y)\ \sin(\pi n z)
\end{eqnarray*}

The analytic solution is
\[
  u(x,y,z,~t) \ =\ sin(\pi l x)\ \sin(\pi m y)\ \sin(\pi n z)\ \exp \left(-(l^2+m^2+n^2)\pi^2t \right)
\]


\section{Discretisation}
%        ==============

Using an equally spaced cartesian grid, with $I\!=\!J\!=\!K$ points in
the $x$, $y$ and $z$ dimensions:
%
\[
  \Delta x \!=\! \Delta y \!=\! \Delta z = \frac{1}{I-1}
\]

Using a time-explicit finite difference scheme -- FTCS (forward in
time, centered in space) -- leads to the discretisation:
%
\[
  u^{n+1}_{i,j,k} = u^n_{i,j,k} + \Delta t \begin{array}[t]{cc}
                                          & \left( \frac{u^n_{i+1,j,k}-2u^n_{i,j,k}+u^n_{i-1,j,k}}{\Delta x^2} \right. \\
                                        + & \frac{u^n_{i,j+1,k}-2u^n_{i,j,k}+u^n_{i,j-1,k}}{\Delta y^2} \\
                                        + & \left. \frac{u^n_{i,j,k+1}-2u^n_{i,j,k}+u^n_{i,j,k-1}}{\Delta z^2} \right)
                                      \end{array}
\]
%
where $u^n_{ijk}$ is the numerical solution at $x_i\!=\!i\Delta x$,
$y_j\!=\!j\Delta y$, $z_k\!=\!k\Delta z$ and at time $t_n\!=\!n\Delta
t$.

The stability condition for the FTCS scheme above is
\[
  \nu \ =\ \fr{\Delta t}{\Delta x^2} \ \leq\ \fr{1}{2*\text{\# dims}}\ =\ \fr{1}{6}
\]

Note: for 2D, the stability condition becomes
\[
  \nu \ =\ \fr{\Delta t}{\Delta x^2} \ \leq\ \fr{1}{4}
\]

\section{Implementation}
%        ==============

Several files available: Matlab, C, C and MPI.  (Fortran solution may be necessary.)

\subsection{Program}
%           -------
Several observations concerning the coding of the explicit scheme above:
%
\begin{itemize}
  \item 3D arrays in C allocated as contiguous linear memory space.
    In accessing entries of the arrays there is a translation from the
    3D indices {\tt i}, {\tt j}, {\tt k} to a linear memory index {\tt
      s}, calculated such that storage is always column major.
  \item Due to the explicit nature of the scheme, the ``new'' solution
    is updated from the ``old'' at each time step.  This means working
    with tho arrays of identical size.  Swapping pointers to two fixed
    memory allocation is preferable to copying values.
\end{itemize}

\subsection{Validation}
%           ----------
Several validation tools:
%
\begin{itemize}
  \item Validate the output of the codes against a baseline Matlab
    implementation.  The output from the codes is a simple ASCII file.
    The MPI code outputs one file per process, which are later merged
    with the command {\tt make merge}.
  \item Validate the scheme implementation with a simple convergence
    test.  The figure below shows the variation of the RMS error
    (continuous line) and the absolute solution error at one fixed
    point in the domain with the number of discretisation points in
    one dimension.  Errors are measured against the analytic solution.
    %
    \begin{figure}
      \centering
      \epsfig{figure = 3D-heat_-_solution-convergence.eps, width  = 0.4\textwidth}
    \end{figure}

\end{itemize}


\subsection{Performance}
%           -----------
















\section{Numerical Solution}
%        ==================

Define the discretisation matrix $A$

\[
  A \ =\
    \left(
      \begin{array}{ccccccccc}
        -4 &  1 &        &        & 1 &   &        &        &    \\
         1 & -4 &    1   &        &   & 1 &        &        &    \\
           &  1 &   -4   & \ddots &   &   & \ddots &        &    \\
           &    & \ddots & \ddots &   &   &        &    1   &    \\
         1 &    &        &        &   &   &        &        &  1 \\
           &  1 &        &        &   &   &        &        &    \\
           &    & \ddots &        &   &   & \ddots & \ddots &    \\
           &    &        &   1    &   &   & \ddots &   -4   &  1 \\
           &    &        &        & 1 &   &        &    1   & -4
      \end{array}
    \right)
\]

Explicit methods:
\[
  \mathbf{u}^{n+1} \ =\ \left( I + \nu A \right)\ \mathbf{u}^n
\]

Implicit methods:
\[
  \left( I - \nu A \right)\ \mathbf{u}^{n+1} \ =\ \mathbf{u}^n
\]

Crank--Nicolson:
\[
  \left( I - \fr{1}{2}\nu A \right)\ \mathbf{u}^{n+1} \ =\
  \left( I + \fr{1}{2}\nu A \right)\ \mathbf{u}^n
\]



\end{document}
